\documentclass[letterpaper, 12pt]{article}
\usepackage[margin=1in]{geometry} % Customize paper layout.
\usepackage{parskip}              % Add space between paragraphs.
\usepackage{enumitem}             % Customize list enumeration.
\usepackage{amsmath, amssymb}     % Extend equation environments.
\usepackage{threeparttable}       % Extend table environment.
\usepackage{longtable}            % Multi-page tables.
\usepackage{siunitx}              % Typeset values with units.
\usepackage{multirow}             % Merge table rows.
\usepackage{graphicx}             % Support figures.
\usepackage{float}                % Place figures exactly.
\usepackage{adjustbox}            % Scale and rotate tables and figures.
\usepackage{rotating}             % Rotate tables.
\usepackage{caption}              % Customize table and figure captions.
\usepackage{color, soul}          % Support highlighting.
\usepackage{listings, courier}    % Extend verbatim environment.
\usepackage{xcolor}               % Make custom colors.

% Write the title block.
\title{Economic downturns and endogenous \\coursetaking in high school}
\author{Patrick Lavallee Delgado}
\date{13 May 2024}

\begin{document}

% Format table values.
\sisetup{
  detect-mode,
  group-digits            = integer,
  group-minimum-digits    = 4,
  group-separator         = {,},
  round-mode              = places,
  round-precision         = 3,
  round-pad               = false,
  input-signs             = ,
  input-symbols           = ,
  input-open-uncertainty  = ,
  input-close-uncertainty = ,
  table-align-text-before = false,
  table-align-text-after  = false,
}

% Remove space above equations.
\setlength{\abovedisplayskip}{0 em}

% Define background color for code blocks.
\definecolor{light-gray}{gray}{0.95}

\maketitle

\section{Motivation}

Policy, research, and first-hand experience have long established that taking more classes and getting good grades lead to better outcomes from high school. For these reasons, schools differentiate their course offerings to engage different types of students. Career and technical education (CTE) is a complement to traditional schooling that introduces students to workplace competencies through hands-on learning within a particular career pathway. High school CTE in particular has been shown to reengage students, decrease drop out, and improve postsecondary outcomes. Students endogenously select into CTE coursework, perhaps as a function of the availability of CTE programs and unobservable preferences for applied learning and workforce readiness. However, students who may not have otherwise considered CTE coursework may be receptive to economic signals that increase their propensity for CTE enrollment, especially because CTE strengthens the connection between school and jobs. For example, a highly salient economic downturn may have students in particularly affected areas make different preparations in high school for a job or college upon graduation. These behavioral changes may include increased CTE coursetaking that in turn result in stronger academic outcomes. However, this means that our understanding of the relationship between coursetaking and outcomes is not as clear as we would think. This work considers potentially endogenous coursetaking in high school in the context of the subprime mortgage crisis of 2007-10 to reexamine its effect on high school completion and college enrollment.

\section{Data}

This study uses two sources of data. Student data are from the High School Longitudinal Survey (HSLS:09), a nationally representative survey administered by the National Center for Education Statistics that follows students from their first year of high school in 2009 through their secondary and postsecondary experiences. HSLS:09 includes student-level information from administrative records and surveys with students, their parents, and their math and science teachers. It also includes school-level data from publicly available sources and surveys with school administrators. We use baseline information from the initial survey when these students were in ninth grade in 2009, and from the second follow-up survey when most would have graduated from high school and possibly enrolled in college in 2016. Importantly, these data provide indicators for high school completion and college enrollment, summary measures of high school coursetaking, and characteristics that are associated with differences in both our outcomes and predictors of interest. The baseline sample contains 25,206 students in 944 high schools; the third follow-up sample retains 12,957 students in 938 high schools.

Table \ref{table:student} summarizes our sample over the characteristics and outcomes we include in the analysis, weighted by students' sampling probability. Panel A lists the outcomes of interest. The high school drop out rate is 17\%, but the on-time graduation rate is 90\%. This is because some students drop out and later return to school; the permanent high school drop out rate is 4\%. The majority of students apply to college and nearly two-thirds enroll in college. In this context, ``college'' includes any two-year or four-year postsecondary institution. Panel B lists the predictors of high school outcomes that we believe are endogenous. These include the credit load and cumulative GPA in academic and CTE courses. One credit unit is equivalent to a year-long course and GPA is on the traditional four-point scale. As expected, students take many fewer CTE courses because they are elective and the average GPA in CTE courses is roughly half a letter grade higher than that in academic courses. Panels C, D, and E show the rich set of student-level and school-level characteristics in our model. Of note are the high expectations for students' educational attainment, with about three-quarters of students and parents reporting aspirations for at least a bachelor's degree. These are the best measures of student and parent motivation in HSLS:09, and conditioning on them will be important to understanding the added effect of coursetaking to high school outcomes.

{\small
\begin{longtable}{lSSSSSSS}
  \caption{Student outcomes and characteristics\label{table:student}} \\
  \hline\hline
  & \multicolumn{1}{c}{Missing} & \multicolumn{1}{c}{Mean} & \multicolumn{1}{c}{Min.} & \multicolumn{1}{c}{25\%} & \multicolumn{1}{c}{50\%} & \multicolumn{1}{c}{75\%} & \multicolumn{1}{c}{Max.} \\
  \hline
  \endhead \\ [-0.5 em]
  & \multicolumn{7}{c}{A: Outcomes} \\ [0.5 em]
  \cline{2-8} \\ [-0.5 em]
  Ever dropped out & 0.00 & 0.17 & 0.00 & 0.00 & 0.00 & 0.00 & 1.00 \\
  Graduated on-time & 0.01 & 0.90 & 0.00 & 1.00 & 1.00 & 1.00 & 1.00 \\
  Applied to college & 0.02 & 0.87 & 0.00 & 1.00 & 1.00 & 1.00 & 1.00 \\
  Enrolled in college & 0.00 & 0.73 & 0.00 & 0.00 & 1.00 & 1.00 & 1.00 \\ [0.5 em]
  \hline \\ [-0.5 em]
  & \multicolumn{7}{c}{B: Endogenous predictors} \\ [0.5 em]
  \cline{2-8} \\ [-0.5 em]
  Credits: academic courses & 0.04 & 18.24 & 0.00 & 16.00 & 19.00 & 22.00 & 53.00 \\
  Credits: CTE courses & 0.04 & 2.94 & 0.00 & 1.00 & 2.50 & 4.00 & 19.00 \\
  GPA: academic courses & 0.05 & 2.56 & 0.00 & 1.96 & 2.61 & 3.24 & 4.00 \\
  GPA: CTE courses & 0.14 & 2.99 & 0.00 & 2.49 & 3.20 & 3.75 & 4.00 \\ [0.5 em]
  \hline \\ [-0.5 em]
  & \multicolumn{7}{c}{C: Student demographics} \\ [0.5 em]
  \cline{2-8} \\ [-0.5 em]
  Female & 0.00 & 0.50 & 0.00 & 0.00 & 0.00 & 1.00 & 1.00 \\
  Race \\
  \hspace{1em} White & 0.00 & 0.52 & 0.00 & 0.00 & 1.00 & 1.00 & 1.00 \\
  \hspace{1em} Hispanic & 0.00 & 0.22 & 0.00 & 0.00 & 0.00 & 0.00 & 1.00 \\
  \hspace{1em} Black & 0.00 & 0.14 & 0.00 & 0.00 & 0.00 & 0.00 & 1.00 \\
  \hspace{1em} Asian & 0.00 & 0.04 & 0.00 & 0.00 & 0.00 & 0.00 & 1.00 \\
  \hspace{1em} Other & 0.00 & 0.09 & 0.00 & 0.00 & 0.00 & 0.00 & 1.00 \\
  Special education status & 0.06 & 0.09 & 0.00 & 0.00 & 0.00 & 0.00 & 1.00 \\
  Socioeconomic status & 0.00 & -0.09 & -1.75 & -0.66 & -0.18 & 0.42 & 2.57 \\
  Poverty status & 0.00 & 0.20 & 0.00 & 0.00 & 0.00 & 0.00 & 1.00 \\
  Parents' educational attainment \\
  \hspace{1em} Less than high school & 0.00 & 0.08 & 0.00 & 0.00 & 0.00 & 0.00 & 1.00 \\
  \hspace{1em} High school diploma or GED & 0.00 & 0.40 & 0.00 & 0.00 & 0.00 & 1.00 & 1.00 \\
  \hspace{1em} Associate's degree & 0.00 & 0.17 & 0.00 & 0.00 & 0.00 & 0.00 & 1.00 \\
  \hspace{1em} Bachelor's degree & 0.00 & 0.21 & 0.00 & 0.00 & 0.00 & 0.00 & 1.00 \\
  \hspace{1em} Graduate degree & 0.00 & 0.14 & 0.00 & 0.00 & 0.00 & 0.00 & 1.00 \\
  Family arrangement \\
  \hspace{1em} Two parents & 0.00 & 0.56 & 0.00 & 0.00 & 1.00 & 1.00 & 1.00 \\
  \hspace{1em} One parent & 0.00 & 0.23 & 0.00 & 0.00 & 0.00 & 0.00 & 1.00 \\
  \hspace{1em} Other & 0.00 & 0.21 & 0.00 & 0.00 & 0.00 & 0.00 & 1.00 \\
  Older siblings \\
  \hspace{1em} Only child & 0.05 & 0.37 & 0.00 & 0.00 & 0.00 & 1.00 & 1.00 \\
  \hspace{1em} One older sibling & 0.05 & 0.33 & 0.00 & 0.00 & 0.00 & 1.00 & 1.00 \\
  \hspace{1em} Two older siblings & 0.05 & 0.16 & 0.00 & 0.00 & 0.00 & 0.00 & 1.00 \\
  \hspace{1em} Three or more older siblings & 0.05 & 0.14 & 0.00 & 0.00 & 0.00 & 0.00 & 1.00 \\ [0.5 em]
  \hline \\ [-0.5 em]
  & \multicolumn{7}{c}{D: Baseline characteristics } \\ [0.5 em]
  \cline{2-8} \\ [-0.5 em]
  Student's educational expectations \\
  \hspace{1em} Less than high school & 0.21 & 0.01 & 0.00 & 0.00 & 0.00 & 0.00 & 1.00 \\
  \hspace{1em} High school diploma or GED & 0.21 & 0.18 & 0.00 & 0.00 & 0.00 & 0.00 & 1.00 \\
  \hspace{1em} Associate's degree & 0.21 & 0.08 & 0.00 & 0.00 & 0.00 & 0.00 & 1.00 \\
  \hspace{1em} Bachelor's degree & 0.21 & 0.24 & 0.00 & 0.00 & 0.00 & 0.00 & 1.00 \\
  \hspace{1em} Graduate degree & 0.21 & 0.50 & 0.00 & 0.00 & 0.00 & 1.00 & 1.00 \\
  Parents' educational expectations \\
  \hspace{1em} Less than high school & 0.10 & 0.00 & 0.00 & 0.00 & 0.00 & 0.00 & 1.00 \\
  \hspace{1em} High school diploma or GED & 0.10 & 0.12 & 0.00 & 0.00 & 0.00 & 0.00 & 1.00 \\
  \hspace{1em} Associate's degree & 0.10 & 0.11 & 0.00 & 0.00 & 0.00 & 0.00 & 1.00 \\
  \hspace{1em} Bachelor's degree & 0.10 & 0.33 & 0.00 & 0.00 & 0.00 & 1.00 & 1.00 \\
  \hspace{1em} Graduate degree & 0.10 & 0.45 & 0.00 & 0.00 & 0.00 & 1.00 & 1.00 \\
  Math class in eighth grade \\
  \hspace{1em} Grade 8 math & 0.02 & 0.23 & 0.00 & 0.00 & 0.00 & 0.00 & 1.00 \\
  \hspace{1em} Pre-Algebra & 0.02 & 0.34 & 0.00 & 0.00 & 0.00 & 1.00 & 1.00 \\
  \hspace{1em} Algebra I & 0.02 & 0.33 & 0.00 & 0.00 & 0.00 & 1.00 & 1.00 \\
  \hspace{1em} Advanced math & 0.02 & 0.06 & 0.00 & 0.00 & 0.00 & 0.00 & 1.00 \\
  \hspace{1em} Other & 0.02 & 0.04 & 0.00 & 0.00 & 0.00 & 0.00 & 1.00 \\
  Math ability & 0.01 & 51.00 & 24.10 & 44.64 & 50.71 & 58.07 & 82.19 \\
  Math self-efficacy & 0.11 & 0.06 & -2.92 & -0.56 & 0.10 & 0.55 & 1.62 \\
  Sense of engagement in school & 0.03 & 0.05 & -3.38 & -0.58 & 0.21 & 0.71 & 1.39 \\
  Sense of belonging at school & 0.04 & 0.03 & -4.35 & -0.39 & -0.04 & 0.77 & 1.59 \\
  Time in extracurriculars \\
  \hspace{1em} Less than one hour & 0.04 & 0.34 & 0.00 & 0.00 & 0.00 & 1.00 & 1.00 \\
  \hspace{1em} One to two hours & 0.04 & 0.23 & 0.00 & 0.00 & 0.00 & 0.00 & 1.00 \\
  \hspace{1em} Two to three hours & 0.04 & 0.23 & 0.00 & 0.00 & 0.00 & 0.00 & 1.00 \\
  \hspace{1em} Three or more hours & 0.04 & 0.20 & 0.00 & 0.00 & 0.00 & 0.00 & 1.00 \\ [0.5 em]
  \hline \\ [-0.5 em]
  & \multicolumn{7}{c}{E: School characteristics } \\ [0.5 em]
  \cline{2-8} \\ [-0.5 em]
  Private school & 0.00 & 0.07 & 0.00 & 0.00 & 0.00 & 0.00 & 1.00 \\
  Locale \\
  \hspace{1em} City & 0.00 & 0.32 & 0.00 & 0.00 & 0.00 & 1.00 & 1.00 \\
  \hspace{1em} Suburb & 0.00 & 0.33 & 0.00 & 0.00 & 0.00 & 1.00 & 1.00 \\
  \hspace{1em} Town & 0.00 & 0.12 & 0.00 & 0.00 & 0.00 & 0.00 & 1.00 \\
  \hspace{1em} Rural & 0.00 & 0.23 & 0.00 & 0.00 & 0.00 & 0.00 & 1.00 \\
  Region \\
  \hspace{1em} Northeast & 0.00 & 0.17 & 0.00 & 0.00 & 0.00 & 0.00 & 1.00 \\
  \hspace{1em} Midwest & 0.00 & 0.22 & 0.00 & 0.00 & 0.00 & 0.00 & 1.00 \\
  \hspace{1em} South & 0.00 & 0.38 & 0.00 & 0.00 & 0.00 & 1.00 & 1.00 \\
  \hspace{1em} West & 0.00 & 0.23 & 0.00 & 0.00 & 0.00 & 0.00 & 1.00 \\
  Percent White & 0.12 & 0.65 & 0.00 & 0.45 & 0.73 & 0.91 & 1.00 \\
  Percent Hispanic & 0.12 & 0.14 & 0.00 & 0.02 & 0.05 & 0.18 & 1.00 \\
  Percent Black & 0.12 & 0.17 & 0.00 & 0.02 & 0.07 & 0.24 & 1.00 \\
  Percent Asian & 0.12 & 0.04 & 0.00 & 0.01 & 0.02 & 0.04 & 1.00 \\
  Percent English language learners & 0.08 & 0.06 & 0.00 & 0.01 & 0.02 & 0.08 & 0.76 \\
  Percent special education & 0.08 & 0.13 & 0.00 & 0.09 & 0.12 & 0.15 & 0.94 \\
  Percent free or reduced price lunch & 0.08 & 0.39 & 0.00 & 0.20 & 0.37 & 0.58 & 1.00 \\
  Climate & 0.20 & -0.55 & -4.22 & -1.24 & -0.41 & 0.22 & 1.97 \\
  Under-resourced & 0.16 & 0.22 & 0.00 & 0.00 & 0.00 & 0.00 & 1.00 \\ [0.5 em]
  \hline \\ [-0.5 em]
  Sample size & 12957 \\ [0.5 em]
  \hline
\end{longtable}
}

Economic data are from the Neighborhood Stabilization Program (NSP) at the US Department of Housing and Urban Development (HUD). The NSP was one of many provisions of the Housing and Economic Recovery Act of 2008, legislated in response to the subprime mortgage crisis. The NSP received an initial allocation of \$3.92\textsc{b} to help state and local governments redevelop abandoned and foreclosed homes, prioritizing communities with the greatest need. The statute defined need in terms of the number and percentage of homes in foreclosure, financed by subprime mortgages, and in default. HUD decided it would be too costly to collect and assemble these data and in a short amount of time. Instead, it used state-level foreclosure rates from the Mortgage Bankers Association National Delinquency Survey with county-level measures from publicly available data to estimate county-level foreclosures rates. HUD chose three predictors: percent decline in home values from peak values in 2000 as of June 2008 from the Office of Federal Housing Enterprise Oversight, percentage of high-cost loans made between 2004 and 2006 from the Federal Reserve Home Mortgage Disclosure Act, and the unemployment rate from the Bureau of Labor Statistics. The NSP data include foreclose rate estimates for the 18 months between January 2007 and June 2008 as well as the predictors that inform those estimates.

Table \ref{table:counties} summarizes average economic conditions between January 2007 and June 2008 at the county level for the counties represented in the HSLS:09 sample. The average foreclosure rate that HUD estimates is 5\%. The distribution suggests significant bunching around the mean between the 25th and 75th percentiles, but there are more counties with relatively lower foreclosure rates than there are with relatively higher foreclosure rates. The average high-cost mortgage rate is 27\%, but there is a long tail of counties with much higher high-cost mortgage rates. The majority of counties did not realize a decline in housing prices relative to peak values in 2000, but those that did saw severe declines and pull the mean below zero. The average unemployment rate is 6\%, and while at least half the sample of counties is near the average, we see some variation in the tails of the distribution.

\begin{table}[!htbp]
  \centering
  \caption{Average county economic conditions from January 2007 to June 2008}
  \label{table:counties}
  \begin{tabular}{lSSSSSSS}
    \hline\hline
    & \multicolumn{1}{c}{Missing} & \multicolumn{1}{c}{Mean} & \multicolumn{1}{c}{Min.} & \multicolumn{1}{c}{25\%} & \multicolumn{1}{c}{50\%} & \multicolumn{1}{c}{75\%} & \multicolumn{1}{c}{Max.} \\
    \hline \\ [-0.5 em]
    Foreclosure rate & 0.00 & 0.05 & 0.00 & 0.03 & 0.05 & 0.06 & 0.12 \\
    High-cost mortgage rate & 0.00 & 0.27 & 0.10 & 0.21 & 0.26 & 0.31 & 0.65 \\
    Housing price decline rate & 0.00 & -0.02 & -0.37 & -0.01 & 0.00 & 0.00 & 0.00 \\
    Unemployment rate & 0.00 & 0.06 & 0.02 & 0.05 & 0.06 & 0.07 & 0.13 \\ [0.5 em]
    \hline \\ [-0.5 em]
    Sample size & 574 \\ [0.5 em]
    \hline
  \end{tabular}
\end{table}

\section{Empirical strategy}

We aim to reestimate the partial effect of high school coursetaking on high school completion and college enrollment. Existing research suggests a strong relationship whereby more courses and better grades increase the probability of high school completion and college enrollment. However, we hypothesize that highly salient economic downturns govern students' selection into additional classes and motivation for better grades, but are themselves unrelated to whether they finish high school and continue onto college, conditional on other student-level and school-level characteristics. Were this hypothesis to hold, we would expect that our current understanding of the relationship between coursetaking and outcomes to contain bias. For example, we could imagine that students respond to negative economic signals by planning earlier for work or college after high school and altering the bundle of academic and CTE courses they take. At the same time, students may be directly affected by a bad economy through parents' reduced earnings or lost employment, which in turn could weaken the supports at home that enable them to succeed in school. In other words, it is plausible that high school coursetaking is an endogenous predictor of high school outcomes. This motivates an instrumental variables approach, whereby we partial out the variation in coursetaking related to economic conditions and reestimate their partial effects on outcomes without that source of potential bias.

\begin{table}[!htbp]
  \centering
  \caption{Correlations between endogenous predictors and instrumental variables}
  \label{table:corr}
  \begin{tabular}{lSSSS}
    \hline\hline
    & \multicolumn{4}{c}{Endogenous predictors} \\
    \cline{2-5}
    & \multicolumn{1}{c}{\multirow{2}{5 em}{\centering Academic \\credits}} & \multicolumn{1}{c}{\multirow{2}{5 em}{\centering CTE \\credits}} & \multicolumn{1}{c}{\multirow{2}{5 em}{\centering Academic \\GPA}} & \multicolumn{1}{c}{\multirow{2}{5 em}{\centering CTE \\GPA}} \\
    Instruments \\
    \hline \\ [-0.5 em]
    Foreclosure rate & -0.1141 & -0.0482 & -0.0720 & -0.0778 \\
    High-cost mortgage rate & -0.0570 & 0.1120 & -0.1209 & -0.1111 \\
    Housing price decline rate & 0.0893 & 0.1944 & 0.0262 & 0.0235 \\
    Unemployment rate & -0.1153 & -0.0317 & -0.0513 & -0.0568 \\ [0.5 em]
    \hline
  \end{tabular}
\end{table}

A first order concern is whether our measures of economic conditions are valid instrumental variables. Specifically, they must be partially correlated with the coursetaking conditional on the other explanatory variables, and they must only affect outcomes through coursetaking. Table \ref{table:corr} shows all pairwise correlations between the endogenous predictors and instrumental variables, weighted by students' sampling probability. Of course, this is just a first step to demonstrating partial correlation as it does not subtract the effects attributable to the student-level and school-level characteristics. Still, we may be concerned that the instruments are weak since the strengths of these relationships are relatively modest. We find support for the exclusion restriction in our results.

Our model takes the form

\begin{align}
  y_{ij} = \hat{\mathbf{G}}_{ij}\boldsymbol{\delta} + \mathbf{X}_{ij} + \boldsymbol{\gamma}_{s} + \varepsilon_{ij}
\end{align}

where $y_{ij}$ is an outcome for student $i$ in school $j$, $\hat{\mathbf{G}}_{ij}$ is a vector of instrumented predictors, $\boldsymbol{\delta}$ is a vector of partial effects of the instrumented predictors and the coefficients of interest, $\mathbf{X}_{ij}$ is a vector of student-level and school-level characteristics, $\boldsymbol{\gamma}_{s}$ is a state fixed effect, and $\varepsilon_{ij}$ is the idiosyncratic error term. Our outcomes are binary measures, so we would interpret the coefficients as the partial effect of a unit increase in an explanatory variable on the probability of realizing the outcome. Of course, this assumes constant rate of change and does not impose any constraint to the unit interval. We estimate the model using system two-stage least squares and a within-state transformation. This allows us to simultaneously handle multiple endogenous predictors with multiple instrumental variables. It also reduces the number of parameters to estimate by eliminating the unobserved heterogeneity within states. Other work prefers a school fixed effect to resolve unobserved differences within schools. However, our instrumental variables are county-level measures, which would fall out in the first stage. Also, a school fixed effect results in a large loss of sample where there is no variation in the outcome, which table \ref{table:student} demonstrates is especially problematic for dropout and on-time graduation. We cluster standard errors at the school level, which is the unit of random sampling of students.

Missingness due to non-response is a problem in the HSLS:09 data, and this is evident in the missingness rates presented in table \ref{table:student}. We use chained multiple imputation to estimate missing values in student-level and school-level characteristics, but we do not impute any outcome. Specifically, we generate 20 datasets through chained linear regression of continuous variables and ordered logit regression of categorical variables where all variables in imputation serve as explanatory variables. The estimates we present are averages across imputed datasets. We developed Stata tools specific to estimation with instrumental variables on multiply imputed data, which are available on the project repository.

\section{Results}

Tables \ref{table:hs_droput}, \ref{table:hs_ontime}, \ref{table:coll_app}, and \ref{table:coll_enr} present our results for dropout, on-time graduation, college application, and college enrollment, respectively. In all tables, column (1) is a regression on the endogenous predictors, column (2) adds the instrumental variables, and column (3) implements our model with system two-stage least squares (2SLS). The penultimate row in each table is the p-value of an F-test on whether the coefficients of interest are together statistically different from zero. In columns (1) and (3), these are the endogenous and instrumented measures of coursetaking; in column (2), these are the instrumental variables. The last row is the sample size available for the outcome; each imputed dataset has the same number of observations. All regressions include the student-level and school-level characteristics in table \ref{table:student} as covariates.

For all outcomes, the regressions in column (1) suggests that additional classes and better grades are associated with positive outcomes, which supports the existing literature. The coefficients generally indicate a small percentage point change in the probability of the outcomes for a unit increase in credits or GPA. Notably, a letter grade increase in academic course GPA is associated with a 7\% increase in the probability of applying to college and a 14\% increase in the probability of enrolling in college. In all regressions, the result of an F-test suggests that these coefficients are together statistically different from zero. The regressions in column (2) tests the exclusion restriction of our instrumental variables. Including these variables in the model does not change the partial effects we estimate for the endogenous predictors. In all regressions, the result of an F-test suggests that these coefficients are together not statistically different from zero. Together, this provides evidence for the validity of our instrumental variables.

The regressions in column (3) are both surprising and consistent: each suggests that high school coursetaking has no effect on high school completion or college enrollment after controlling for endogenous selection and motivation related to economic conditions. The point estimates on the instrumented measures of coursetaking change in both size and sign, and their standard errors increase substantially. None is statistically different from zero; we only fail to reject the null hypothesis that they are together different from zero for the regression of dropout. We might be concerned that the instrumental variables procedure is driving these null results by increasing our standard errors. If that were true, we would need to develop a theory that reconciles unexpected changes in the direction of the effects. For example, ignoring statistical significance, it seems unlikely that additional CTE classes would be associated with increased dropout and decreased college application, and yet better CTE grades would have the opposite effect for the same outcomes. It also seems unlikely that CTE coursetaking would be associated with decreased on-time graduation since we do not see that students enroll in enough CTE credits to displace required academic

\begin{table}[!htbp]
  \resizebox{\textwidth}{!}{
    \begin{threeparttable}
      \caption{Endogenous coursetaking: ever dropped out of high school}
      \label{table:hs_droput}
      \begin{tabular}{lSSS}
        \hline\hline
        & \multicolumn{1}{c}{(1)} & \multicolumn{1}{c}{(2)} & \multicolumn{1}{c}{(3)} \\
        & \multicolumn{1}{c}{OLS} & \multicolumn{1}{c}{OLS} & \multicolumn{1}{c}{2SLS} \\
        \hline \\
        \input{out/hs_dropout.tex} \\ [-0.5 em]
        \hline
      \end{tabular}
      \begin{tablenotes}
        \small
        \item \textit{Notes:} Robust standard errors clustered at the high school level in parentheses. F-tests in columns (1) and (3) test whether the coefficients of interest are together statistically different from zero. F-test in column (2) tests whether the proposed instrumental variables are together statistically different from zero.
        \item $^{*}\text{p}<0.05$; $^{**}\text{p}<0.01$; $^{***}\text{p}<0.001$
      \end{tablenotes}
    \end{threeparttable}
  }
\end{table}

\begin{table}[!htbp]
  \resizebox{\textwidth}{!}{
    \begin{threeparttable}
      \caption{Endogenous coursetaking: graduated from high school on-time}
      \label{table:hs_ontime}
      \begin{tabular}{lSSS}
        \hline\hline
        & \multicolumn{1}{c}{(1)} & \multicolumn{1}{c}{(2)} & \multicolumn{1}{c}{(3)} \\
        & \multicolumn{1}{c}{OLS} & \multicolumn{1}{c}{OLS} & \multicolumn{1}{c}{2SLS} \\
        \hline \\
        \input{out/hs_ontime.tex} \\ [-0.5 em]
        \hline
      \end{tabular}
      \begin{tablenotes}
        \small
        \item \textit{Notes:} Robust standard errors clustered at the high school level in parentheses. F-tests in columns (1) and (3) test whether the coefficients of interest are together statistically different from zero. F-test in column (2) tests whether the proposed instrumental variables are together statistically different from zero.
        \item $^{*}\text{p}<0.05$; $^{**}\text{p}<0.01$; $^{***}\text{p}<0.001$
      \end{tablenotes}
    \end{threeparttable}
  }
\end{table}

\begin{table}[!htbp]
  \resizebox{\textwidth}{!}{
    \begin{threeparttable}
      \caption{Endogenous coursetaking: applied to college}
      \label{table:coll_app}
      \begin{tabular}{lSSS}
        \hline\hline
        & \multicolumn{1}{c}{(1)} & \multicolumn{1}{c}{(2)} & \multicolumn{1}{c}{(3)} \\
        & \multicolumn{1}{c}{OLS} & \multicolumn{1}{c}{OLS} & \multicolumn{1}{c}{2SLS} \\
        \hline \\
        \input{out/coll_app.tex} \\ [-0.5 em]
        \hline
      \end{tabular}
      \begin{tablenotes}
        \small
        \item \textit{Notes:} Robust standard errors clustered at the high school level in parentheses. F-tests in columns (1) and (3) test whether the coefficients of interest are together statistically different from zero. F-test in column (2) tests whether the proposed instrumental variables are together statistically different from zero.
        \item $^{*}\text{p}<0.05$; $^{**}\text{p}<0.01$; $^{***}\text{p}<0.001$
      \end{tablenotes}
    \end{threeparttable}
  }
\end{table}

\begin{table}[!htbp]
  \resizebox{\textwidth}{!}{
    \begin{threeparttable}
      \caption{Endogenous coursetaking: enrolled in college}
      \label{table:coll_enr}
      \begin{tabular}{lSSS}
        \hline\hline
        & \multicolumn{1}{c}{(1)} & \multicolumn{1}{c}{(2)} & \multicolumn{1}{c}{(3)} \\
        & \multicolumn{1}{c}{OLS} & \multicolumn{1}{c}{OLS} & \multicolumn{1}{c}{2SLS} \\
        \hline \\
        \input{out/coll_enr.tex} \\ [-0.5 em]
        \hline
      \end{tabular}
      \begin{tablenotes}
        \small
        \item \textit{Notes:} Robust standard errors clustered at the high school level in parentheses. F-tests in columns (1) and (3) test whether the coefficients of interest are together statistically different from zero. F-test in column (2) tests whether the proposed instrumental variables are together statistically different from zero.
        \item $^{*}\text{p}<0.05$; $^{**}\text{p}<0.01$; $^{***}\text{p}<0.001$
      \end{tablenotes}
    \end{threeparttable}
  }
\end{table}

\section{Discussion}

The present analysis finds that controlling for measures of the subprime mortgage crisis of 2007-10 for students who entered high school in 2009 eliminates differences in high school completion and college enrollment as they relate to high school coursetaking. This demonstration of endogenous coursetaking threatens the causal claim that basic elements of the education production function -- taking more classes and getting good grades -- directly lead to better secondary and postsecondary outcomes. It also suggests that highly salient economic downturns may help explain differential selection into additional classes and motivation for better grades. This has broad implications for education policy: perhaps schools are not the right place to intervene where the barriers to success are at home. This also has immediate implications for the literature: this question needs further study to support a major departure from existing research.

\end{document}
